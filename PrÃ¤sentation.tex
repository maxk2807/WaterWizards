\documentclass{beamer}

\title{WaterWizards}
\author{Justin Dewitz, Max Kondratov, Erick Zeiler, Julian <Nachname>}
\date{17. Juli 2025}

\begin{document}

\frame{\titlepage}

\begin{frame}
\frametitle{Inhaltsverzeichnis}
\tableofcontents
\end{frame}

\section{Demo + Einleitung}
\begin{frame}
\frametitle{Demo + Einleitung}
\begin{itemize}
  \item Was ist WaterWizards?
  \begin{enumerate}
    \item Mulitplayer Real-Time Schiffe versenken
    \item Angriff durch Zauber, die durch Karten repräsentiert werden
    \item Ziel: Zerstörung der gegnerischen Schiffe
  \end{enumerate}
  \item Warum WaterWizards?
  \begin{enumerate}
    \item Schiffe versenken ist ein Klassiker
    \item Durch Real-Time wird es dynamischer
    \item Für jede Altersgruppe interessant
  \end{enumerate}
\end{itemize}
\end{frame}

\section{Rollen des Projektes}
\begin{frame}
\frametitle{Rollen des Projektes}

\end{frame}

\subsection{Ansprechpartner}
\begin{frame}
\frametitle{Ansprechpartner}

\end{frame}

\section{Technologie + Architektur}
\begin{frame}
\frametitle{Technologie + Architektur}
  \begin{itemize}
    \item Raylib für das Rendern des Spiels auf dem Client
    \item LiteNetLib für die Client-Server-Verbindung
  \end{itemize}
\end{frame}



\subsection{UI}
\begin{frame}
  \frametitle{UI}
\end{frame}

\subsection{GameScreen}
\begin{frame}
\frametitle{GameScreen}
  GameScreen enthält alles, was auf dem Bildschirm gezeigt wird:
  \begin{enumerate}
    \item Die beiden Spielbretter (GameBoard) mit Schiffen (GameShip) und Steinen
    \item Die Karten (GameCard) auf der Hand (GameHand) der Spieler
    \item Die Kartenstapel (CardStack) der einzelnen Kartentypen (CardType)
  \end{enumerate}
\end{frame}


\subsection{Server/Backend}
\begin{frame}
\frametitle{Server/Backend}
\end{frame}

\subsection{Shared}
\begin{frame}
\frametitle{Shared}
\end{frame}

\subsection{Backend - Client Kommunukation}
\begin{frame}
\frametitle{Backend - Client Kommunikation}
\end{frame}

\section{Wiki}
\begin{frame}
\frametitle{Wiki}

\end{frame}

\section{Erfahrungen und Fazit}
\begin{frame}
\frametitle{Erfahrungen und Fazit}

\end{frame}

\end{document}